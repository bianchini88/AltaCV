%%%%%%%%%%%%%%%%%
% This is an sample CV template created using altacv.cls
% (v1.6.4, 13 Nov 2021) written by LianTze Lim (liantze@gmail.com). Now compiles with pdfLaTeX, XeLaTeX and LuaLaTeX.
%
%% It may be distributed and/or modified under the
%% conditions of the LaTeX Project Public License, either version 1.3
%% of this license or (at your option) any later version.
%% The latest version of this license is in
%%    http://www.latex-project.org/lppl.txt
%% and version 1.3 or later is part of all distributions of LaTeX
%% version 2003/12/01 or later.
%%%%%%%%%%%%%%%%

%% Use the "normalphoto" option if you want a normal photo instead of cropped to a circle
% \documentclass[10pt,a4paper,normalphoto]{altacv}

\documentclass[10pt,a4paper,ragged2e,withhyper]{altacv}
%% AltaCV uses the fontawesome5 and packages.
%% See http://texdoc.net/pkg/fontawesome5 for full list of symbols.

% Change the page layout if you need to
\geometry{left=1.25cm,right=1.25cm,top=1.5cm,bottom=1.5cm,columnsep=1.2cm}

% The paracol package lets you typeset columns of text in parallel
\usepackage{paracol}

% Adding this for flexibility
%\usepackage{hyperref}
%\usepackage[colorlinks]{hyperref}
\usepackage{academicons}
\usepackage{xcolor}


% Change the font if you want to, depending on whether
% you're using pdflatex or xelatex/lualatex
\ifxetexorluatex
  % If using xelatex or lualatex:
  \setmainfont{Roboto Slab}
  \setsansfont{Lato}
  \renewcommand{\familydefault}{\sfdefault}
\else
  % If using pdflatex:
  \usepackage[rm]{roboto}
  \usepackage[defaultsans]{lato}
  % \usepackage{sourcesanspro}
  \renewcommand{\familydefault}{\sfdefault}
\fi

% Change the colours if you want to
\definecolor{SlateGrey}{HTML}{2E2E2E}
\definecolor{LightGrey}{HTML}{666666}
\definecolor{DarkPastelRed}{HTML}{450808}
\definecolor{PastelRed}{HTML}{8F0D0D}
\definecolor{GoldenEarth}{HTML}{E7D192}
\colorlet{name}{black}
\colorlet{tagline}{PastelRed}
\colorlet{heading}{DarkPastelRed}
\colorlet{headingrule}{GoldenEarth}
\colorlet{subheading}{PastelRed}
\colorlet{accent}{PastelRed}
\colorlet{emphasis}{SlateGrey}
\colorlet{body}{LightGrey}

% Change some fonts, if necessary
\renewcommand{\namefont}{\Huge\rmfamily\bfseries}
\renewcommand{\personalinfofont}{\footnotesize}
\renewcommand{\cvsectionfont}{\LARGE\rmfamily\bfseries}
\renewcommand{\cvsubsectionfont}{\large\bfseries}


% Change the bullets for itemize and rating marker
% for \cvskill if you want to
\renewcommand{\itemmarker}{{\small\textbullet}}
\renewcommand{\ratingmarker}{\faCircle}

%% Use (and optionally edit if necessary) this .cfg if you
%% want to use an author-year reference style like APA(6)
%% for your publication list
\input{pubs-authoryear.cfg}

%% Use (and optionally edit if necessary) this .cfg if you
%% want an originally numerical reference style like IEEE
%% for your publication list
% \input{pubs-num.cfg}

%% sample.bib contains your publications
\addbibresource{bianchini.bib}

\begin{document}
\name{Federico Bianchini}
\tagline{Principle Engineer / Data Steward}
%% You can add multiple photos on the left or right
%\photoR{2.8cm}{Globe_High}
%\photoR{2.5cm}{bilde.jpg}
% \photoL{2.5cm}{Yacht_High,Suitcase_High}

\personalinfo{%
  % Not all of these are required!
  \orcid{0000-0002-9016-4820}
  \email{fredebi@uio.no}
  \printinfo{\faGlobe}{home}[https://www.mn.uio.no/kjemi/english/people/aca/]
  \github{bianchini88}
  \printinfo{~\aiGoogleScholar~}{Scholar}[https://scholar.google.com/citations?user=K92o8rcAAAAJ&hl=en]
  \printinfo{~\aiPublons~}{Publons}[https://publons.com/wos-op/researcher/1706853/federico-bianchini/]
  \printinfo{\faLinkedin}{00700381}[https://www.linkedin.com/in/federico-bianchini-00700381/]

  %% You can add your own arbitrary detail with
  %% \printinfo{symbol}{detail}[optional hyperlink prefix]
  % \printinfo{\faPaw}{Hey ho!}[https://example.com/]
  %% Or you can declare your own field with
  %% \NewInfoFiled{fieldname}{symbol}[optional hyperlink prefix] and use it:
  % \NewInfoField{gitlab}{\faGitlab}[https://gitlab.com/]
  % \gitlab{your_id}
  %%
  %% For services and platforms like Mastodon where there isn't a
  %% straightforward relation between the user ID/nickname and the hyperlink,
  %% you can use \printinfo directly e.g.
  % \printinfo{\faMastodon}{@username@instace}[https://instance.url/@username]
  %% But if you absolutely want to create new dedicated info fields for
  %% such platforms, then use \NewInfoField* with a star:
  % \NewInfoField*{mastodon}{\faMastodon}
  %% then you can use \mastodon, with TWO arguments where the 2nd argument is
  %% the full hyperlink.
  % \mastodon{@username@instance}{https://instance.url/@username}
}

\makecvheader
%% Depending on your tastes, you may want to make fonts of itemize environments slightly smaller
% \AtBeginEnvironment{itemize}{\small}

%% Set the left/right column width ratio to 6:4.
\columnratio{0.6}

% Start a 2-column paracol. Both the left and right columns will automatically
% break across pages if things get too long.
\begin{paracol}{2}
\cvsection{Experience}

\cvevent{}
        {Principle Engineer -- Centre for Bioinformatics}
        {October 2020 -- Ongoing}
        {University of Oslo, Norway}
%\begin{itemize}
%\item FAIR data management in the Life Sciences
%\item Job description 2
%\end{itemize}

        \cvevent{}
                {Researcher -- Department of Chemistry}
                {April 2020 -- September 2020}
                {University of Oslo, Norway}
%\begin{itemize}
%\item Atomistic modelling of materials -- nuclear magnetic resonance
%\end{itemize}

%\divider

                \cvevent{}{Postdoctoral Fellow -- Department of Chemistry}
                        {April 2020 -- September 2020}
                        {University of Oslo, Norway}
%\begin{itemize}
%\item Atomistic modelling of ionic mobility in battery materials
%\end{itemize}

\cvsection{Education}

\cvevent{}
        {Ph.D.\ in Physics -- King's College London}
        {October 2012 -- August 2016}
        {London, UK}
%\textbf{Thesis}: ``Mechanical Properties of Nickel-based Superalloys A Multiscale Atomistic Investigation''\\
%\textbf{Supervisor}: Prof. Alessandro De Vita

\cvevent{}
        {M.Sc.\ in Solid State Physics -- University of Trieste}
        {October 2010 -- September 2012}
        {Trieste, Italy}
%\textbf{Thesis}: ``Graphene Growth on Ni(111) Surface: a First Principle Study''\\
%\textbf{Supervisor}: Prof: Maria Peressi\\
%\textbf{110/110 \textit{Summa cum laude}}

\cvevent{}
        {B.Sc.\ in Physics -- University of Trieste}
        {October 2007 -- September 2010}
        {Trieste, Italy}


\cvsection{Events}

\cvevent{}{ELIXIR All-Hands 2021}{1--11 June 2021}{Virtual}

\cvevent{}{Operando surface catalysis meeting}{Jan./Feb. 2019}{Oslo, Norway}

\cvevent{}{Modelling and simulation of superalloys}{March 2017}{Bochum, Germany}

\cvevent{}{FAIR data management in the life sciences}{14--16 June 2022}{Virtual}

\cvevent{}{Carpentry@UiO: Plotting and Programming with Python (Novices)}{1 June 2022}{University of Oslo}

\cvevent{}{Nordic Train the Trainer course}{10-13 May 2022}{Virtual}

\cvevent{}{Data Management Planning workshop for Life Science projects}{4-5 April 2022}{Virtual}





%\cvsection{A Day of My Life}

% Adapted from @Jake's answer from http://tex.stackexchange.com/a/82729/226
% \wheelchart{outer radius}{inner radius}{
% comma-separated list of value/text width/color/detail}
%\wheelchart{1.5cm}{0.5cm}{%
%  6/8em/accent!30/{Sleep,\\beautiful sleep},
%  3/8em/accent!40/Hopeful novelist by night,
%  8/8em/accent!60/Daytime job,
%  2/10em/accent/Sports and relaxation,
%  5/6em/accent!20/Spending time with family
%}

% use ONLY \newpage if you want to force a page break for
% ONLY the current column


\cvsection{Publications}

\nocite{*}

%\printbibliography[heading=pubtype,title={\printinfo{\faBook}{Books}},type=book]

%\divider

\printbibliography[heading=pubtype,title={\printinfo{\faFile*[regular]}{Journal Articles (selected)}},type=article]

\printbibliography[heading=pubtype,title={\printinfo{\faCheck}{Project deliverables}},type=misc]







%% Switch to the right column. This will now automatically move to the second
%% page if the content is too long.
\switchcolumn

%\cvsection{My Life Philosophy}

%\begin{quote}
%``Something smart or heartfelt, preferably in one sentence.''
%\end{quote}

%\cvsection{Most Proud of}

%\cvachievement{\faTrophy}{Fantastic Achievement}{and some details about it}

%\divider

%\cvachievement{\faHeartbeat}{Another achievement}{more details about it of course}

%\divider

%\cvachievement{\faHeartbeat}{Another achievement}{more details about it of course}

\cvsection{Strengths}

\cvtag{Data management planning}
\cvtag{FAIR data}\\
\smallskip

\cvtag{modelling}
\cvtag{batteries}
\cvtag{catalysis}
\\
\smallskip

\cvtag{python}
\cvtag{Fortran}
\cvtag{\TeX}
\cvtag{git}


\cvsection{Projects}

\cveventbis{}
        {ELIXIR Norway}
        {April 2022--ongoing}
        {
          \href{https://elixir.no/}
               {elixir.no}
        }

\cveventbis{}
        {BioMedData}
        {June 2020–ongoing}
        {
          \href{https://elixir.no/organization/biomeddata}
               {homepage}
        }

\cveventbis{}
        {ELIXIR CONVERGE}
        {Feb. 2020–Ongoing}
        {
          \href{https://elixir-europe.org/about-us/how-funded/eu-projects/converge}
               {homepage}
        }


\cvsection{Community}

\cveventtris{ELIXIR-related groups and projects:}
        {
          \begin{itemize}
          \item ELIXIR Norway
          \item RDMkit editorial board
          \item ELIXIR CONVERGE
          \end{itemize}
        }
        {}{}


\cveventtris{EOSC task force:}
        {
          \begin{itemize}
          \item Data Stewardship, curricula and career paths
          \end{itemize}
        }
        {}{}

\cveventtris{University of Oslo:}
        {
          \begin{itemize}
          \item Carpentry@UiO (helper)
          \item UiO Data Managers Network (board)
          \end{itemize}
        }
        {}{}
        



%% Yeah I didn't spend too much time making all the
%% spacing consistent... sorry. Use \smallskip, \medskip,
%% \bigskip, \vspace etc to make adjustments.
%\medskip


\cvsection{Languages}

\cvtag{Italian}
\cvtag{English}
\cvtag{Norwegian}




\cvsection{Reference}

% \cvref{name}{email}{mailing address}
\cvref{Prof.\ Eivind Hovig}{Centre for Bioinformatics,\\
  ~~University of Oslo, Norway}{ehovig@ifi.uio.no}{}


\end{paracol}


\end{document}
